% \CheckSum{489}
% \iffalse meta-comment
% 
% This is file `alnumsec.dtx'.
% 
% Copyright (C) 2003-2007 Frank K�ster (frank@kuesterei.ch)
% 
% --------------------------------------------------------------------------
% 
% This work may be distributed and/or modified under the
% conditions of the LaTeX Project Public License, either version 1.3
% of this license or (at your option) any later version.
% The latest version of this license is in
%   http://www.latex-project.org/lppl.txt
% and version 1.3 or later is part of all distributions of LaTeX
% version 2003/12/01 or later.
% 
% This work has the LPPL maintenance status "maintained".
% 
% This Current Maintainer of this work is Frank K�ster.
% 
% This work consists of the files README, alnumsec.ins, alnumsec.dtx,
% and the derived file alnumsec.sty.
% 
% \fi
%
% \iffalse
% \charactertable
%  {Upper-case    \A\B\C\D\E\F\G\H\I\J\K\L\M\N\O\P\Q\R\S\T\U\V\W\X\Y\Z
%   Lower-case    \a\b\c\d\e\f\g\h\i\j\k\l\m\n\o\p\q\r\s\t\u\v\w\x\y\z
%   Digits        \0\1\2\3\4\5\6\7\8\9
%   Exclamation   \!     Double quote  \"     Hash (number) \#
%   Dollar        \$     Percent       \%     Ampersand     \&
%   Acute accent  \'     Left paren    \(     Right paren   \)
%   Asterisk      \*     Plus          \+     Comma         \,
%   Minus         \-     Point         \.     Solidus       \/
%   Colon         \:     Semicolon     \;     Less than     \<
%   Equals        \=     Greater than  \>     Question mark \?
%   Commercial at \@     Left bracket  \[     Backslash     \\
%   Right bracket \]     Circumflex    \^     Underscore    \_
%   Grave accent  \`     Left brace    \{     Vertical bar  \|
%   Right brace   \}     Tilde         \~}
% \fi
%
% \iffalse
%<*dtx>
            \ProvidesFile{alnumsec.dtx}
%</dtx>
%\NeedsTeXFormat{LaTeX2e}[1995/12/01]
%<driver>\ProvidesFile{alnumsec.drv}
%<alnumsec>\ProvidesPackage{alnumsec}
            [2007/07/27 v0.03 alphanumeric section numbers with
            ordinary sectioning commands]
%<*driver>
\documentclass{ltxdoc}
\usepackage{hypdoc}
\hypersetup{pdfkeywords={LaTeX,package,alnumsec},pdfstartview={}}
\EnableCrossrefs
\DoNotIndex{\@addtoreset,\@arabic,\@gobble,\@Alph,\@alph}
\DoNotIndex{\@ctrerr,\@ifnextchar,\@ifstar,\@ifundefined}
\DoNotIndex{\@minus,\@namedef,\@nameuse,\@nil}
\DoNotIndex{\@onelevel@sanitize,\@plus,\@sptoken}
\DoNotIndex{\@startsection}
\DoNotIndex{\\,\ ,\|,\@tmpA,\@tmpB,\@tmpC}
\DoNotIndex{\addtocounter,\alpha}
\DoNotIndex{\AfterPackage,\arabic,\AtEndDocument,\AtBeginDocument,\author,\addchap}
\DoNotIndex{\beta,\begin,\caption,\ClassError,\ClassWarning,\csname,\CurrentOption}
\DoNotIndex{\c@experiment,\cfigure,\c@fk@argnumber,\c@paragraph,\c@subexperiment,\c@subparagraph,\c@subsection}
\DoNotIndex{\centering,\chapter,\chi}
\DoNotIndex{\DeclareOption,\def,\documentclass,\dots,\date,\delta}
\DoNotIndex{\edef,\emph,\endinput,\epsilon,\eta}
\DoNotIndex{\if,\else,\fi,\empty,\end,\endcsname,\expandafter}
\DoNotIndex{\fbox,\FK@@sp@b,\FK@@sp@c,\FK@@sp@d,\FK@tempa,\FK@toks@,\FK@@sp@def,\frontmatter,\futurelet}
\DoNotIndex{\fk@old@addcontentsline,\fk@old@appendix,\fk@old@backmatter}
\DoNotIndex{\if@mainmatter,\fi,\ifnum,\fi}
\DoNotIndex{\ifx,\fi,\input,\ifcase,\or,\fi,\iota}
\DoNotIndex{\gamma,\kappa,\lambda}
\DoNotIndex{\l@chapter,\l@section,\l@subsection,\l@experiment,\l@labday,\l@subexperiment,\label,\let,\LoadClass}
\DoNotIndex{\maketitle,\MessageBreak,\mu}
\DoNotIndex{\newcommand,\newcounter,\newif,\newtoks,\nobreak,\normalfont,\nu}
\DoNotIndex{\othertext,\omega}
\DoNotIndex{\par,\PassOptionsToClass,\printindex,\ProcessOptions,\PackageError,\pi,\psi}
\DoNotIndex{\protect,\protected@edef,\protected@xdef,\providecommand}
\DoNotIndex{\raggedsection,\refstepcounter,\relax,\renewcommand,\RequirePackage,\rule,\rho}
\DoNotIndex{\@roman,\@Roman}
\DoNotIndex{\sectfont,\setcounter,\size@section,\space,\stepcounter,\sigma}
\DoNotIndex{\tableofcontents,\tau}
\DoNotIndex{\title,\upsilon,\value,\varepsilon,\varphi,\vartheta}
\DoNotIndex{\usepackage,\Xothertext,\Xsometext,\z@,\xi,\zeta}
\CodelineIndex
%
\begin{document}
 \DocInput{alnumsec.dtx}
\end{document}
%</driver>
% \fi
% 
% \GetFileInfo{alnumsec.dtx}
% \RecordChanges
%
% \MacroIndent=2em
% \setcounter{IndexColumns}{2}
% 
% \MakeShortVerb{\|}
% \MakeShortVerb{\�}
%
% \title{alnumsec.sty: Using alphanumeric section numbering with
%   standard sectioning commands\texorpdfstring{\thanks{%^^A
%          This package has version number \fileversion, last revised \filedate.}}{}}
% \author{Frank K\"uster\\
%         \href{mailto:frank@kuesterei.ch}{\texttt{frank@kuesterei.ch}}}
% \date{\filedate}
% \maketitle{}
%
% \begin{abstract}
%   \noindent This package allows you to use alphanumeric section 
%   numbering, e.g.  A. Introduction; III. International Law. It's
%   output is similar to alphanum.sty, but you can use the standard
%   \LaTeX\ sectioning commands. Thus it is possible to switch numbering
%   schemes easily.  Greek letters, double letters (bb) and different
%   delimiters around them are supported.
% \end{abstract}
%
% \tableofcontents{}
% \def\partname{Part}
%
% \clearpage
% \part{User documentation}
% \label{part:user-documentation}
%
% \section{Options}
% \label{sec:options}
%
% There is only one option: If you specify |usehighlevels|, then the
% numbers for every heading will start with the numbers of the superior
% levels, as without the package (e.g. A.II.3.(a) Important Section).
% Without the option, only the number of the current section level is
% used, e.g. (a) Important Section. This seems to be common with
% alphanumeric numbering.
%
% \section{Specifying the numbering scheme}
% \label{sec:spec-numb-scheme}
%
% You specify the numbering scheme for the headings with the macro
% \begin{verbatim}
% \alnumsecstyle{<list of one-letter-specifiers>}
% \end{verbatim}
% with the following specifiers:
% \begin{center}
% \begin{minipage}{0.8\textwidth}\renewcommand*{\thempfootnote}{\arabic{mpfootnote}}
% \begin{description}
% \item[a or n] stands for \textbf{arabic} number: 6
% \item[R] stands for an uppercase \textbf{Roman} number: VI
% \item[r] stands for a lowercase \textbf{roman} number: vi
% \item[L] stands for an uppercase \textbf{Letter}: A
% \item[l] stands for a lowercase \textbf{letter}: a
% \item[g] is for \textbf{greek} letter: $\alpha$
% \item[d] is for two lowercase letters (\textbf{doubleletter}): aa, bb
% \item[b] for two greek letters (\textbf{doublegreek})\footnotemark: $\alpha\alpha$
% \end{description}
% \end{minipage}
% \end{center}
% \footnotetext{The b is from \emph{bis}, latin for "twice", since the
%   greek word would also yield a \textbf{d}.}
%
% The numbers and letters are all followed by a period per default, if
% you want to change this, use the macro
% \begin{verbatim}
% \surround<name>{<before>}{<after>}
% \end{verbatim}
% Instead of |<name>|, put the word given in boldface in the list above
% (case matters!), |<before>| and |<after>| will be typeset around the
% number of type name.
%
% If you use some number types twice, e.g. arabic numbers for the second
% and fifth level: |\alnumsecstyle{LaRla}|, you might want to
% distinguish between the two numbers by their separators. To achieve
% this, you give the separators for lower levels in the optional
% argument(s) to |\surround<name>|. In the example, you might use
% |\surroundarabic[(]{}{)}| to achieve A. 2) III. d. (4) or
% |\surroundarabic[(][)]{}{.}| for A. 2. III. d. (4). Note, however,
% that alnumsec does \emph{not} recognize that a number type has yet
% been used and that it now should use the alternative separators.
% Instead, you have to specify the first level for which alternative
% separators should be used with the command
% |\otherseparators{<level>}|. In \LaTeX, chapter, if defined, has
% level 0, section has 1 and so on down to subparagraph with level
% 5. Therefore, in the above example, |\otherseparators{5}| would
% work, but |\otherseparators{3}| as well.
%
% Please note that |\alnumsecstyle| does not change the numbering
% scheme of figures or tables (yet).
% So if a |report| or |book| like document class will be used,
% one can get double periods there.
% But this can be corrected easily, for example the |figure| counter
% representation is usually defined as something like
% \begin{quote}
%  |\newcommand\thefigure{%|\\
%  |  \ifnum\value{chapter}>0 \thechapter.\fi \arabic{figure}}|
% \end{quote}
% So to remove the extra period here one can insert
% \begin{quote}
%  |\renewcommand\thefigure{%|\\
%  |  \ifnum\value{chapter}>0 \thechapter\fi \arabic{figure}}|
% \end{quote}
% right after the use of |\alnumsecstyle|. (same for tables)
%
%
% \section{Sectioning levels to use}
% \label{sec:sect-levels-use}
%
% Many people that use alphanumeric sectioning numbers also seem to use
% many, many levels of sectioning commands - alnumsec.sty can handle
% this. If you only use the levels that are defined in the standard
% classes (i.e. from |\chapter| or |\section| to |\subparagraph|), you
% don't have to do anything. |\part| is not treated at all by
% alnumsec.sty.
%
%\newcommand*{\labday}{\textbackslash\texttt{labday}}\newcommand*{\experiment}{\textbackslash\texttt{experiment}}
%\newcommand*{\typesection}{\textbackslash\texttt{section}}\newcommand*{\typechapter}{\textbackslash\texttt{chapter}}
% If you have more (or simply other\footnote{With other, I mean other
%   names for the same concept -- e.g. my labbook.cls uses \labday\
%   instead of \typechapter\ and \experiment\ instead of \typesection.
%   Different concepts, as e.g. in alphanum.sty, won't work. But anyway,
%   you'll only want to use one of both.}) sectioning macros, you have
% to tell alnumsec about their names and whether the first is on
% \LaTeX{}s level 0 (like chapter) or 1 (like section). This is done
% with the macro |\alnumsectionlevels| - here is what the package uses
% for the article class:
% \begin{verbatim}
%  \alnumsectionlevels{1}{section,subsection,subsubsection,paragraph,subparagraph}
% \end{verbatim}
%
% If you use this macro, you have to do it \emph{before}
% |\alnumsecstyle|!\smallskip
%
% alnumsec.sty assumes that the number is typeset using |\the<name>| for
% section level |<name>|. This will always be the case if the macro has
% been defined using the \LaTeX\ macro designed for this,
% |\@startsection|.
%
%
% \section{Bugs and Limitations}
% \label{sec:bugs-limitations}
%
% Currently I am not aware of any real bugs, but one could imagine a lot
% of more features. However, since I wrote this package for somebody
% else's needs and don't use it myself, I need input from users to be
% able to improve it.
%
% \StopEventually{\PrintIndex\PrintChanges}
%
% \clearpage
% \part{Implementation}
%
%
%
%    \begin{macrocode}
%<*alnumsec>
\newif\ifusepreviouslevels\usepreviouslevelsfalse%
\DeclareOption{usehighlevels}{\usepreviouslevelstrue}%
\ProcessOptions%
\RequirePackage{ifthen}%
%    \end{macrocode}
% 
% Macros for greek "numbers" and double letters:
% \changes{v0.03}{2007/07/27}{Define \cs{@greek} et al with \cs{providecommand}}
%
%    \begin{macrocode}
\providecommand*{\@greek}[1]{\ifcase#1\relax\or$\alpha$\or$\beta$\or
  $\gamma$\or$\delta$\or$\varepsilon$\or$\zeta$\or$\eta$\or$\vartheta$\or
  $\iota$\or$\kappa$\or$\lambda$\or$\mu$\or$\nu$\or$\xi$\or$o$\or$\pi$\or
  $\rho$\or$\sigma$\or$\tau$\or$\upsilon$\or$\varphi$\or$\chi$\or$\psi$\or
  $\omega$\else\@ctrerr\fi}%
\providecommand*{\@doublegreek}[1]{\@greek{#1}{\@greek{#1}}}
\providecommand*{\@doublealph}[1]{\@alph{#1}{\@alph{#1}}}
\newcounter{alnumsec@level}%
\newcounter{fk@secdepth}%
\newcounter{fk@secstart}%
\newcounter{fk@changelevel}\setcounter{fk@changelevel}{20}%
%    \end{macrocode}
% 
% |alnumsec@level| is the dynamic counter used while browsing through
% the levels. |fk@secdepth| is the number of sectioning levels for
% which names are known and thus numbers can be assigned. |fk@secstart|
% will be the starting value for every use of |alnumsec@level|, i.e. it
% will be 0 if |\chapter|\footnote{or some other macro on the level 0}
% is defined and  1 otherwise. |fk@changelevel| is the level from which
% the alternative separators for lower levels will be used. It is
% initially set very high so that lower level separators won't be used
% unless this counter is changed, using the following command:
%
%    \begin{macrocode}
\def\otherseparators#1{%
  \setcounter{fk@changelevel}{#1}
}
%    \end{macrocode}
% 
% |\alnumsectionlevels| is the command for users that have more or
% different than the usual section names. The main work is done by
% |\fk@countlevels|, after that |fk@secdepth| is set to the number of
% known levels.
%
%    \begin{macrocode}
\def\alnumsectionlevels#1#2{%
  \setcounter{fk@secstart}{#1}
  \setcounter{alnumsec@level}{#1}%
  \fk@countlevels#2,\relax,%
  \setcounter{fk@secdepth}{\value{alnumsec@level}}%
  \addtocounter{fk@secdepth}{-1}
}
%    \end{macrocode}
% 
% |\fk@countlevels| goes through the comma separated list of level names
% until it encounters the relax that has been put at the end by
% |\alnumsectionlevels|. For each level, it puts this name into a
% "numbered" name, e.g. |\fk@levelname1|, and increases the counter.
%
%    \begin{macrocode}
\def\fk@countlevels#1,{%
  \ifx\relax#1%
    \empty%
  \else%
    \expandafter\def\csname fk@levelname\thealnumsec@level\endcsname{#1}%
    \stepcounter{alnumsec@level}%
    \expandafter\fk@countlevels%
  \fi%
}
%    \end{macrocode}
% 
% |\alnumsecstyle| is the macro with which the user specifies the
% numbering scheme and, implicitly, the level of the last numbered
% section. It feeds its argument to |\fk@scanstyle| and later sets
% |secnumdepth|. This counter has to be lowered by one because
% |\fk@scanstyle| increments |alnumsec@level| \emph{after} it has parsed
% each letter, so after the last letter it is incremented once more.
% Then |\fk@assignstyle| is called which actually defines |\thesection|
% and friends.
%
%    \begin{macrocode}
\def\alnumsecstyle#1{%
  \setcounter{alnumsec@level}{\value{fk@secstart}}%
  \fk@scanstyle#1\relax%
  \setcounter{secnumdepth}{\value{alnumsec@level}}%
  \addtocounter{secnumdepth}{-1}%
  \setcounter{alnumsec@level}{\value{fk@secstart}}%
  \fk@assignstyle%
}%
\def\fk@scanstyle#1{%
  \ifx\relax#1%
    \relax%
  \else%
    \ifnum\c@alnumsec@level>\c@fk@secdepth%
      \PackageError{alnumsec}{%
        more numbering levels than sectioning levels}{%
        You have specified \thealnumsec@level\space different
        numbering styles.\MessageBreak However, only
        \thefk@secdepth\space sectioning commands have been defined,
        down to \csname fk@levelname\thefk@secdepth\endcsname.
      }%
    \else%
      \fk@whichstyle{#1}%
      \stepcounter{alnumsec@level}%
    \fi%
    \expandafter\fk@scanstyle%
  \fi%
}
\newif\iffk@letterknown\fk@letterknownfalse
\def\fk@whichstyle#1{%
  \if R#1%
    \fk@defsecstyle{\thealnumsec@level}{\@Roman}{Roman}%
    \fk@letterknowntrue
  \fi%
  \if r#1%
    \fk@defsecstyle{\thealnumsec@level}{\@roman}{roman}%
    \fk@letterknowntrue
  \fi%
  \if n#1%
    \fk@defsecstyle{\thealnumsec@level}{\@arabic}{arabic}%
    \fk@letterknowntrue
  \fi%
  \if a#1%
    \fk@defsecstyle{\thealnumsec@level}{\@arabic}{arabic}%
    \fk@letterknowntrue
  \fi%
  \if L#1%
    \fk@defsecstyle{\thealnumsec@level}{\@Alph}{Letter}%
    \fk@letterknowntrue
  \fi%
  \if l#1%
    \fk@defsecstyle{\thealnumsec@level}{\@alph}{letter}%
    \fk@letterknowntrue
  \fi%
  \if g#1%
    \fk@defsecstyle{\thealnumsec@level}{\@greek}{greek}%
    \fk@letterknowntrue
  \fi%
  \if d#1%
    \fk@defsecstyle{\thealnumsec@level}{\@doublealph}{doubleletter}%
    \fk@letterknowntrue
  \fi%
  \if b#1%
    \fk@defsecstyle{\thealnumsec@level}{\@doublegreek}{doublegreek}%
    \fk@letterknowntrue
  \fi%
  \iffk@letterknown\else%
    \PackageError{alnumsec}{unknown specifier: #1}{%
      You have given #1 as specifier for the numbering
      scheme.\MessageBreak
      Only the following are known:\MessageBreak
      nrRLldgb
    }
  \fi
}
\def\fk@defsecstyle#1#2#3{%
  \edef\fk@seclevel{\csname fk@levelname#1\endcsname}%
  \expandafter\def\csname fk@\thealnumsec@level num\endcsname{#2}%
  \ifnum\c@alnumsec@level<\c@fk@changelevel%
    \expandafter\def%
      \csname fk@pre@\thealnumsec@level\expandafter\endcsname\expandafter{%
      \csname fk@pre@#3\endcsname}%
    \expandafter\def%
      \csname fk@post@\thealnumsec@level\expandafter\endcsname\expandafter{%
      \csname fk@post@#3\endcsname}%
  \else%
    \expandafter\def%
      \csname fk@lower@pre@\thealnumsec@level\expandafter\endcsname\expandafter{%
      \csname fk@lower@pre@#3\endcsname}%
    \expandafter\def%
      \csname fk@lower@post@\thealnumsec@level\expandafter\endcsname\expandafter{%
      \csname fk@lower@post@#3\endcsname}%
  \fi%
}%
%    \end{macrocode}
%
% In |\fk@assignstyle|, the first level has to be treated differently to
% allow the use of the previous levels for the lower levels.
%
%    \begin{macrocode}
\def\fk@assignstyle{%
  \edef\fk@secname{\csname fk@levelname\thefk@secstart\endcsname}%
  \expandafter\@namedef{the\fk@secname\expandafter}\expandafter{%
    \csname fk@pre@\thealnumsec@level\expandafter\endcsname%
    \csname fk@\thefk@secstart num\expandafter\endcsname%
    \csname c@\fk@secname\expandafter\endcsname%
    \csname fk@post@\thealnumsec@level\expandafter\endcsname%
  }%
  \whiledo{%
    \c@alnumsec@level<\c@secnumdepth%
  }{%
    \stepcounter{alnumsec@level}%
    \let\fk@previoussecname\fk@secname%
    \edef\fk@secname{%
      \csname fk@levelname\thealnumsec@level\endcsname}%
    \ifusepreviouslevels%
      \ifnum\c@alnumsec@level<\c@fk@changelevel%
        \expandafter\@namedef{the\fk@secname\expandafter}\expandafter{%
          \csname the\fk@previoussecname\expandafter\endcsname
          \csname fk@pre@\thealnumsec@level\expandafter\endcsname%
          \csname fk@\thealnumsec@level num\expandafter\endcsname%
          \csname c@\fk@secname\expandafter\endcsname%
          \csname fk@post@\thealnumsec@level\endcsname}%
      \else%
        \expandafter\@namedef{the\fk@secname\expandafter}\expandafter{%
          \csname the\fk@previoussecname\expandafter\endcsname
          \csname fk@lower@pre@\thealnumsec@level\expandafter\endcsname%
          \csname fk@\thealnumsec@level num\expandafter\endcsname%
          \csname c@\fk@secname\expandafter\endcsname%
          \csname fk@lower@post@\thealnumsec@level\endcsname}%
      \fi%
    \else%
%    \end{macrocode}
%
% The following three lines are added to have references with parents,
% thanks to Markus Kohm. \changes{v0.02}{2005/02/16}{Added code to
% make references complete, thanks to Markus Kohm.}
%
%    \begin{macrocode}
      \expandafter\@namedef{p@\fk@secname\expandafter}\expandafter{%
        \csname p@\fk@previoussecname\expandafter\endcsname
        \csname the\fk@previoussecname\endcsname}%
      \ifnum\c@alnumsec@level<\c@fk@changelevel%
        \expandafter\@namedef{the\fk@secname\expandafter}\expandafter{%
          \csname fk@pre@\thealnumsec@level\expandafter\endcsname%
          \csname fk@\thealnumsec@level num\expandafter\endcsname%
          \csname c@\fk@secname\expandafter\endcsname%
          \csname fk@post@\thealnumsec@level\endcsname}%
      \else%
        \expandafter\@namedef{the\fk@secname\expandafter}\expandafter{%
          \csname fk@lower@pre@\thealnumsec@level\expandafter\endcsname%
          \csname fk@\thealnumsec@level num\expandafter\endcsname%
          \csname c@\fk@secname\expandafter\endcsname%
          \csname fk@lower@post@\thealnumsec@level\endcsname}%
      \fi%
    \fi%
  }%
}
\def\define@surroundstyle#1{%
  \@namedef{surround#1}{%
    \@ifnextchar [{%]
      \csname opt@surround#1\endcsname}{%
      \csname nopt@surround#1\endcsname}%
  }%
  \@namedef{opt@surround#1}[##1]{%
    \@ifnextchar [{%]
      \csname dopt@surround#1\endcsname[##1]}{%
      \csname @opt@surround#1\endcsname[##1]}
  }
  \@namedef{dopt@surround#1}[##1][##2]##3##4{%
    \@namedef{fk@lower@pre@#1}{##1}%
    \@namedef{fk@lower@post@#1}{##2}%
    \@namedef{fk@pre@#1}{##3}%
    \@namedef{fk@post@#1}{##4}%
  }
  \@namedef{@opt@surround#1}[##1]##2##3{%
    \relax
    \@namedef{fk@lower@pre@#1}{##1}%
    \@namedef{fk@pre@#1}{##2}%
    \@namedef{fk@post@#1}{##3}%
    \expandafter\let%
      \csname fk@lower@post@#1\expandafter\endcsname%
      \csname fk@post@#1\endcsname%
  }
  \@namedef{nopt@surround#1}##1##2{%
    \@namedef{fk@pre@#1}{##1}%
    \@namedef{fk@post@#1}{##2}%
    \expandafter\let%
      \csname fk@lower@pre@#1\expandafter\endcsname%
      \csname fk@pre@#1\endcsname%
    \expandafter\let%
      \csname fk@lower@post@#1\expandafter\endcsname%
      \csname fk@post@#1\endcsname%
  }
}
\define@surroundstyle{Roman}
\define@surroundstyle{roman}
\define@surroundstyle{Letter}
\define@surroundstyle{letter}
\define@surroundstyle{arabic}
\define@surroundstyle{doubleletter}
\define@surroundstyle{greek}
\define@surroundstyle{doublegreek}
\newif\iffk@chapterdefined%
\@ifundefined{chapter}{%
  \fk@chapterdefinedfalse%
  \setcounter{fk@secstart}{1}%
  \setcounter{fk@secdepth}{5}%
  \alnumsectionlevels{1}{section,subsection,subsubsection,paragraph,subparagraph}%
}{%
  \fk@chapterdefinedtrue%
  \setcounter{fk@secstart}{0}%
  \setcounter{fk@secdepth}{5}%
  \alnumsectionlevels{0}{chapter,section,subsection,subsubsection,paragraph,subparagraph}%
}
\iffk@chapterdefined%
  \def\fk@pre@chapter{}%
  \def\fk@post@chapter{.}%
\fi
\surroundRoman{}{.}
\surroundroman{}{.}
\surroundarabic{}{.}
\surroundLetter{}{.}
\surroundletter[(]{}{)}
\surroundgreek[(]{}{)}
\surrounddoubleletter[(]{}{)}
\surrounddoublegreek[(]{}{)}
%</alnumsec>
%    \end{macrocode}
% 
% \clearpage
% \Finale
%
\endinput


\documentclass[12pt]{article}

\setlength{\topmargin}{-.75in} \addtolength{\textheight}{2.00in}
\setlength{\oddsidemargin}{.00in} \addtolength{\textwidth}{.75in}

\usepackage{amsmath,color,graphicx,array,multirow,rotating, enumerate}
\usepackage{type1cm}
\usepackage{eso-pic}
\usepackage[hmargin=2cm,vmargin=1.3cm]{geometry}
\usepackage{mathabx}
\usepackage[rflt]{/Users/jgates/desktop/latex/floatflt}
\usepackage[table]{xcolor}
\nofiles

\def\Tab#1{\tabular[t]{>{\rule[-1ex]{0pt}{3ex}}c}#1\endtabular}
\newcolumntype{C}{@{}c@{}}

\pagestyle{empty}
\newcounter{ProbNum}
\setlength{\parindent}{0in}

% Watermark: graph paper
\newcommand\BackgroundPic{
\put(0,0){
\parbox[b][\paperheight]{\paperwidth}{%
\vfill
\centering
\includegraphics[width=\paperwidth,height=\paperheight,keepaspectratio]{/Users/jgates/desktop/latex/pics/plain.pdf}%
\vfill
}}}

%Diagram box command [v space][content]
\newcommand{\diagrambox}[2][40 mm]{
\framebox{\parbox{175 mm}{#2 \hfill \\ \vspace{#1}}}

\bigskip
}

% MakeList: [example number] [content]
\newcommand{\MakeList}[2]{
\begin{enumerate}[#1] \itemsep1pt \parskip0pt \parsep0pt  

#2
\end{enumerate}
}

\begin{document}



% Watermark
\AddToShipoutPicture*{\BackgroundPic}

\addtocounter {ProbNum} {1}
\centering

% Standards section - start here


{\footnotesize \begin{tabular}{| p{.15 cm}  p{.15 cm} | p{1.7 cm} | p{13 cm} | }
\hline
\multirow{8}{*}
{\rotatebox[origin=c]{90}{\parbox{30 mm}{{\large{\bf  OPM:\phantom{l}G}}}}}  
&\multirow{6}{*}
{\rotatebox[origin=c]{90}{{\parbox{42 mm}{\scriptsize \centering Oscillation Graphs}}}} &Core Skills 	& Calculate the amplitude of an oscillator's motion from a position graph  \\ \cline{4-4}
& & 					& Calculate the period and frequency of an oscillator's motion from a graph  \\ \cline{3-4}			
& & \multirow{2}{*}{\parbox{1.7cm}{Proficiency Indicators}}	& Determine the equilibrium point of an oscillator from a position graph\\ \cline{4-4}
& & 					& Determine the displacement during a time interval from a position or velocity graph \\ \cline{4-4}
& & 					& Determine the velocity of an oscillator at a moment from a position or velocity graph \\ \cline{4-4}
& & 					& Identify driving and damping forces and their effects on oscillation graphs \\ \cline{4-4}
& & 					& Understand and identify maxima, minima, and zeroes of position and velocity from position or velocity graphs \\ \cline{3-4}			
& & Adv. Ind.	& Determine the acceleration from the velocity graph and identify acceleration maxima on position and velocity graphs \\ \hline
 \hline
\end{tabular} }
\vspace{2 mm}



{\footnotesize \begin{tabular}{| p{.15 cm}  p{.15 cm} | p{1.7 cm} | p{13 cm} | }
\hline
\multirow{8}{*}
{\rotatebox[origin=c]{90}{\parbox{26 mm}{{\large{\bf OPM:\phantom{l}T/f }}}}}  
&\multirow{8}{*}
{\rotatebox[origin=c]{90}{{\parbox{34 mm}{\scriptsize \centering Period and Frequency}}}} &Core Skills 	& Calculate the period and frequency of an oscillator from data, including data that you take yourself  \\ \cline{4-4}
& & 					& Know and use the units for period and frequency correctly and consistently  \\ \cline{3-4}						
& & \multirow{2}{*}{\parbox{1.7cm}{Proficiency Indicators}}	& Use period and frequency to solve problems  \\ \cline{4-4}
& &					& Understand the variables that affect the period and frequency of pendula \\ \cline{4-4}
& & 					& Understand the variables that affect the period and frequency of spring systems \\ \cline{3-4}
& & Adv. Ind.	& Use the algebraic models relating the period/frequency of pendula or spring systems to physical variables (mass, spring stiffness, string length) \\ \hline
\end{tabular} }
\vspace{2 mm}



% end here
\end{document}
\documentclass[12pt]{article}

\setlength{\topmargin}{-.75in} \addtolength{\textheight}{2.00in}
\setlength{\oddsidemargin}{.00in} \addtolength{\textwidth}{.75in}

\usepackage{amsmath,color,graphicx, array}
\usepackage{type1cm}
\usepackage{eso-pic}
\usepackage[hmargin=2cm,vmargin=1.3cm]{geometry}
\usepackage{mathabx}
\usepackage[rflt]{/Users/jgates/desktop/latex/floatflt}
\usepackage{enumerate}
\usepackage{multicol}

\nofiles

\pagestyle{empty}

\setlength{\parindent}{0in}

% Watermark: graph paper
\newcommand\BackgroundPic{
\put(0,0){
\parbox[b][\paperheight]{\paperwidth}{%
\vfill
\centering
\includegraphics[width=\paperwidth,height=\paperheight,keepaspectratio]{/Users/jgates/desktop/latex/pics/plain.pdf}%
\vfill
}}}

%Lab Reminder section command
\newcommand{\labreminders}{
\vspace{-5mm} Be sure to: \vspace{-3mm}\begin{itemize} \itemsep1pt \parskip0pt \parsep0pt 
\renewcommand{\labelitemi}{$\rightarrow$}
\item Use pencil
\item Label your axes with symbols and units
\item Give the graph a title (�[vertical axis variable] vs. [horizontal axis variable]�)
\item Draw a best fit line/curve (don't connect the dots).
\item Find the slope using points on the line (not data points).
\item Write the equation of the line using the variables from your axes (don't default to �y and x�); make sure the constants (like slope and intercept) have the correct units attached to the numbers.
\item Put units on numbers, but never on variables
\end{itemize}
}

%Diagram box command [v space][content]
\newcommand{\diagrambox}[2][40 mm]{
\framebox{\parbox{175 mm}{#2 \hfill \\ \vspace{#1}}}

\bigskip
}

% PNum: problem numbering command
\newcounter{ProbNum}
\newcommand{\PNum}{
\addtocounter {ProbNum} {1}
{\bf \Large{\arabic{ProbNum}}.}
}

% MakeList: [example number] [content]
\newcommand{\MakeList}[2]{
\begin{enumerate}[#1] \itemsep1pt \parskip0pt \parsep0pt 
#2
\end{enumerate}
}

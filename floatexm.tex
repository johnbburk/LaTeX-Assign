%%
%% This is file `floatexm.tex',
%% generated with the docstrip utility.
%%
%% The original source files were:
%%
%% floatflt.dtx  (with options: `exempelkod')
%% 
%% Copyright (c) 1994-1998 by Mats Dahlgren <matsd@sssk.se>.
%% All rights reserved.  See the file `floatflt.ins' for information
%% on how you may (re-)distribute the `floatflt' package files.
%% You are not allowed to make any changes to this file without
%% explicit permission from the author.
%% 
\documentclass[11pt]{article}
\usepackage{floatflt}
\begin{document}
\centerline{\Huge The Tale of
\textsf{floatflt}}\bigskip

\noindent This is a demonstration document for the use
of the \textsf{floatflt} package.  It contains several
floating figures and tables with captions explaining how
they were called.  At the end, both a
\verb+\listoffigures+ and a \verb+\listoftables+ command
are used, resulting in the desired lists.  For more
details on how to use the \textsf{floatflt} package,
please run \LaTeX{} on the file \texttt{floatflt.dtx}.

The following work by Edgar Alan Poe was retrieved by
anonymous ftp from \texttt{ftp.funet.fi} in the
directory \texttt{/pub/doc/literary/etext} where it is
found in the file \texttt{telltale.poe}.  Only minor
\LaTeX\ adaptions have been done, besides the inclusion
of floating floats.

\begin{center}
Internet Wiretap Edition of\\[2mm]
{\large THE TELL-TALE HEART}\\
by\\ {\large EDGAR ALLAN POE}\\[2mm]
From \textit{The Works of Edgar Allan Poe: Tales Vol I}\\
J. B. Lippincott Co, Copyright 1895.\\[2mm]
This text is placed into the Public Domain (May 1993).\\
\end{center}

\noindent\textit{\Large The Tell-Tale Heart}

\noindent TRUE! nervous, very, very dreadfully nervous
I had been and am; but why WILL you say
that I am mad? The disease had sharpened
my senses, not destroyed, not dulled them. Above
all was the sense of hearing acute. I heard all things
in the heaven and in the earth. I heard many things
in hell. How then am I mad? Hearken! and observe how healthily, how
calmly, I can tell you the
whole story.

\begin{floatingfigure}{60mm}
\begin{center}
The first figure to\\ use the environment\\ \texttt{floatingfigure}
\end{center}
\caption{The \texttt{floatingfigure} environment with \texttt{60mm}
for \textit{width} and no \textit{option}.}
\end{floatingfigure}
It is impossible to say how first the idea entered
my brain, but, once conceived, it haunted me day
and\linebreak night. Object there was none. Passion there
was none. I loved the old man. He had never
wronged me. He had never given me insult. For
his gold I had no desire. I think it was his eye!
Yes, it was this! One of his eyes resembled that
of a vulture -- a pale blue eye with a film over it.
Whenever it fell upon me my blood ran cold, and
so by degrees, very gradually, I made up my mind
to take the life of the old man, and thus rid myself
of the eye for ever.

\begin{floatingfigure}[r]{40mm}
\begin{center}
Another figure to\\ use the environment\\ \texttt{floatingfigure}
\end{center}
\caption{The \texttt{floatingfigure} environment with \texttt{40mm}
for \textit{width} and the \texttt{r} \textit{option}.}
\end{floatingfigure}
Now this is the point. You fancy me mad. Madmen
know nothing. But you should have seen me.
You should have seen how wisely I proceeded --
with what caution -- with what foresight, with what
dissimulation, I went to work! I was never kinder
to the old man than during the whole week before
I killed him. And every night about midnight I
turned the latch of his door and opened it oh, so
gently! And then, when I had made an opening
sufficient for my head, I put in a dark lantern all
closed, closed so that no light shone out, and then
I thrust in my head. Oh, you would have laughed
to see how cunningly I thrust it in! I moved it
slowly, very, very slowly, so that I might not
disturb the old man's sleep. It took me an hour to
place my whole head within the opening so far that
I could see him as he lay upon his bed. Ha! would
a madman have been so wise as this? And then
when my head was well in the room I undid the
lantern cautiously -- oh, so cautiously -- cautiously
(for the hinges creaked), I undid it just so much
that a single thin ray fell upon the vulture eye.
And this I did for seven long nights, every night
just at midnight, but I found the eye always closed,
and so it was impossible to do the work, for it was
not the old man who vexed me but his Evil Eye.
And every morning, when the day broke, I went
boldly into the chamber and spoke courageously to
him, calling him by name in a hearty tone, and
inquiring how he had passed the night. So you see
he would have been a very profound old man, indeed,
to suspect that every night, just at twelve, I
looked in upon him while he slept.

\begin{floatingtable}{
\begin{tabular}{cccc}
$x$ & $x^2$ & $x^3$ & $x^4$ \\ \hline
1 & 1 & 1 & 1 \\
2 & 4 & 8 & 16 \\
3 & 9 & 27 & 81 \\
\end{tabular}}
\caption{The \texttt{floatingtable} environment with no \textit{option}.}
\end{floatingtable}
Upon the eighth night I was more than usually
cautious in opening the door. A watch's minute
hand moves more quickly than did mine. Never
before that night had I felt the extent of my own
powers, of my sagacity. I could scarcely contain
my feelings of triumph. To think that there I was
opening the door little by little, and he not even to
dream of my secret deeds or thoughts. I fairly
chuckled at the idea, and perhaps he heard me, for
he moved on the bed suddenly as if startled. Now
you may think that I drew back -- but no. His room
was as black as pitch with the thick darkness (for
the shutters were close fastened through fear of
robbers), and so I knew that he could not see the
opening of the door, and I kept pushing it on
steadily, steadily.

\begin{floatingtable}[l]{
\begin{tabular}{ccc}
$\alpha$ & $\sin\alpha$ & $\cos\alpha$ \\ \hline
0 & 0 & 1 \\
$\pi$ & 0 & $-1$ \\
$2\pi$ & 0 & 1 \\
\end{tabular}}
\caption{The \texttt{floatingtable} environment with the \texttt{l}
\textit{option}.}
\end{floatingtable}
I had my head in, and was about to open the
lantern, when my thumb slipped upon the tin fasten\-ing,
and the old man sprang up in the bed, crying
out, ''Who's there?''

I kept quite still and said nothing. For a whole
hour I did not move a muscle, and in the meantime
I did not hear him lie down. He was still sitting
up in the bed, listening; just as I have done night
after night hearkening to the death watches in the
wall.

\begin{floatingfigure}[l]{50mm}
\begin{center}
\Large A Figure!
\end{center}
\caption{The \texttt{floatingfigure} environment
with \textit{width} set to \texttt{50mm}  and the
\texttt{l} \textit{option}.}
\end{floatingfigure}
Presently, I heard a slight groan, and I knew it
was the groan of mortal terror. It was not a groan of
pain or of grief -- oh, no! It was the low stifled sound
that arises from the bottom of the soul when
overcharged with awe. I knew the sound well. Many
a night, just at midnight, when all the world slept,
it has welled up from my own bosom, deepening,
with its dreadful echo, the terrors that distracted
me. I say I knew it well. I knew what the old
man felt, and pitied him although I chuckled at
heart. I knew that he had been lying awake ever
since the first slight noise when he had turned in
the bed. His fears had been ever since growing
upon him. He had been trying to fancy them
causeless, but could not. He had been saying to
himself, ''It is nothing but the wind in the chimney,
it is only a mouse crossing the floor,'' or, ''It is merely
a cricket which has made a single chirp.'' Yes he
has been trying to comfort himself with these
suppositions; but he had found all in vain. ALL IN VAIN,
because Death in approaching him had stalked with
his black shadow before him and enveloped the
victim. And it was the mournful influence of the
unperceived shadow that caused him to feel, although
he neither saw nor heard, to feel the presence
of my head within the room.

When I had waited a long time very patiently
without hearing him lie down, I resolved to open
a little -- a very, very little crevice in the lantern.
So I opened it -- you cannot imagine how stealthily,
stealthily -- until at length a single dim ray like the
thread of the spider shot out from the crevice and
fell upon the vulture eye.

It was open, wide, wide open, and I grew furious
as I gazed upon it. I saw it with perfect distinctness
-- all a dull blue with a hideous veil over it that
chilled the very marrow in my bones, but I could
see nothing else of the old man's face or person, for
I had directed the ray as if by instinct precisely upon
the damned spot.

And now have I not told you that what you mis-
take for madness is but over-acuteness of the senses?
now, I say, there came to my ears a low, dull, quick
sound, such as a watch makes when enveloped in
cotton. I knew that sound well too. It was the
beating of the old man's heart. It increased my fury
as the beating of a drum stimulates the soldier into
courage.

\begin{floatingfigure}[p]{90mm}
\begin{center}
A rather wide figure which still uses the \\
\texttt{floatingfigure} environment.
\end{center}
\caption{A \texttt{floatingfigure} environment
which uses \texttt{90mm} for \textit{width} and
the \texttt{p} \textit{option}.}
\end{floatingfigure}
But even yet I refrained and kept still. I scarcely\linebreak
brea\-thed. I held\linebreak the lantern motionless. I tried
how\linebreak stead\-ily I could\linebreak
maintain the ray upon the eye.
Meantime the hellish tattoo of the heart increased.
It grew quicker and quicker, and louder and louder,
every instant. The old man's terror must have
been extreme! It grew louder, I say, louder every
moment! -- do you mark me well? I have told you
that I am nervous: so I am. And now at the dead
hour of the night, amid the dreadful silence of that
old house, so strange a noise as this excited me to
uncontrollable terror. Yet, for some minutes longer
I refrained and stood still. But the beating grew
louder, louder! I thought the heart must burst.
And now a new anxiety seized me -- the sound would
be heard by a neighbour! The old man's hour had
come! With a loud yell, I threw open the lantern
and leaped into the room. He shrieked once -- once
only. In an instant I dragged him to the floor, and
pulled the heavy bed over him. I then smiled
gaily, to find the deed so far done. But for many
minutes the heart beat on with a muffled sound.
This, however, did not vex me; it would not be
heard through the wall. At length it ceased. The
old man was dead. I removed the bed and examined
the corpse. Yes, he was stone, stone dead. I placed
my hand upon the heart and held it there many
minutes. There was no pulsation. He was stone
dead. His eye would trouble me no more.

\begin{floatingtable}[r]{
\begin{tabular}{l}
Two lines in one table with one single \\
column is enough!
\end{tabular}}
\caption{This \texttt{floatingtable}  uses the \texttt{r}
\textit{option}.}
\end{floatingtable}
If still you think me mad, you will think so no
longer when I describe the wise precautions I took
for the concealment of the body. The night waned,
and I worked hastily, but in silence.

I took up three planks from the flooring of the
chamber, and deposited all between the scantlings.
I then replaced the boards so cleverly so cunningly,
that no human eye -- not even his -- could have
detected anything wrong. There was nothing to wash
out -- no stain of any kind -- no blood-spot whatever.
I had been too wary for that.

When I had made an end of these labours, it was
four o'clock -- still dark as midnight. As the bell
sounded the hour, there came a knocking at the
street door. I went down to open it with a light
heart, -- for what had I now to fear? There entered
three men, who introduced themselves, with perfect
suavity, as officers of the police. A shriek had been
heard by a neighbour during the night; suspicion
of foul play had been aroused; information had been
lodged at the police office, and they (the officers)
had been deputed to search the premises.

\begin{floatingtable}[p]{
\begin{tabular}{|l|l|l|} \hline
English Word & Swedish Word & Dutch Word \\ \hline
read & l\"asa & lesen \\
speak & tala & spreken\\
write & skriva & schrijven \\ \hline
\end{tabular}}
\caption{A \texttt{floatingtable}  with the \texttt{p}
\textit{option}.}
\end{floatingtable}
I smiled, -- for what had I to fear? I bade the
gentlemen welcome. The shriek, I said, was my
own in a dream. The old man, I mentioned,\linebreak was
absent in the country. I took my visitors all over
the house. I bade them search -- search well. I led
them, at length, to his chamber. I showed them his
treasures, secure, undisturbed. In the enthusiasm
of my confidence, I brought chairs into the room,
and desired them here to rest from their fatigues,
while I myself, in the wild audacity of my perfect
triumph, placed my own seat upon the very spot
beneath which reposed the corpse of the victim.

The officers were satisfied. My MANNER had
convinced them. I was singularly at ease. They sat
and while I answered cheerily, they chatted of
familiar things. But, ere long, I felt myself getting
pale and wished them gone. My head ached, and
I fancied a ringing in my ears; but still they sat,
and still chatted. The ringing became more distinct:
I talked more freely to get rid of the feeling:
but it continued and gained definitiveness -- until,
at length, I found that the noise was NOT within my
ears.

No doubt I now grew VERY pale; but I talked
more fluently, and with a heightened voice. Yet
the sound increased -- and what could I do? It was
A LOW, DULL, QUICK SOUND -- MUCH SUCH A SOUND AS A
WATCH MAKES WHEN ENVELOPED IN COTTON. I gasped for
breath, and yet the officers heard it not. I talked
more quickly, more vehemently but the noise
steadily increased. I arose and argued about trifles,
in a high key and with violent gesticulations; but
the noise steadily increased. Why WOULD  they not
be gone? I paced the floor to and fro with heavy
strides, as if excited to fury by the observations of
the men, but the noise steadily increased. O God!
what COULD I do? I foamed -- I raved -- I swore! I
swung the chair upon which I had been sitting, and
grated it upon the boards, but the noise arose over
all and continually increased. It grew louder --
louder -- louder! And still the men chatted
pleasantly, and smiled. Was it possible they heard
not? Almighty God! -- no, no? They heard! --
they suspected! -- they KNEW! -- they were making
a mockery of my horror! -- this I thought, and this
I think. But anything was better than this agony!
Anything was more tolerable than this derision! I
could bear those hypocritical smiles no longer! I
felt that I must scream or die! -- and now -- again
-- hark! louder! louder! louder! LOUDER! --

''Villains!'' I shrieked, ''dissemble no more! I
admit the deed! -- tear up the planks! -- here, here!
-- it is the beating of his hideous heart!''

\rightline{\ensuremath{\mathcal{END}}.}

\listoffigures
\listoftables

\end{document}
\endinput
%%
%% End of file `floatexm.tex'.
